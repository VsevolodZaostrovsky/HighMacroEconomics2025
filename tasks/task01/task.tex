\section{Вопрос 1. Что такое абсолютная или условная конвергенция? Какая конвергенция наблюдается? Какую конвергенцию предсказывает модель Солоу‑Свана? Почему?}

\subsection*{Контекст}
В неоклассической модели Солоу‑Свана динамика капитала на эффективного работника определяется уравнением
\[
\dot k = s f(k) - (n + g + \delta)\,k,
\]
где \(k\) — капитал на эффективного работника, \(s\) — норма сбережений, \(n\) — темп роста населения, \(g\) — темп
 технологического прогресса, \(\delta\) — норма амортизации. 
 Производственная функция \(f(k)\) обладает убывающей отдачей от капитала.

\subsection*{Ответ}
\begin{itemize}
  \item \textbf{Абсолютная конвергенция} означает, что экономики с более низким уровнем дохода на душу населения растут быстрее независимо от их фундаментальных характеристик.
  \item \textbf{Условная конвергенция} означает, что более бедные экономики растут быстрее только при прочих равных (одинаковые нормы сбережений, демография и технологии).
  \item \textbf{Эмпирические данные} показывают наличие \emph{условной} конвергенции между странами.
  \item \textbf{Модель Солоу‑Свана} предсказывает \emph{условную} конвергенцию, поскольку каждая экономика сходится к 
  собственному стационарному уровню капитала \(k^*\), который зависит от параметров \(s\), \(n\), \(g\) и \(\delta\). 
  Расстояние до этого стационарного уровня определяет скорость роста: чем дальше текущее \(k\) от \(k^*\), тем выше темп 
  роста дохода на душу населения {\it (стр.~10)}.
\end{itemize}
