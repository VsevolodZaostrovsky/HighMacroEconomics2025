\section*{Вопрос 2. Какие предположения (и почему) лежат в основе модели Солоу‑Свана?}

\subsection*{Контекст}
Модель Солоу‑Свана является базовой неоклассической моделью экономического роста. Её ключевой элемент — производственная функция, 
зачастую представляемая в виде функции Cobb–Douglas:
\[
Y = K^\alpha (AL)^{1-\alpha},
\]
где \(Y\) — совокупный выпуск, \(K\) — капитал, \(L\) — труд, \(A\) — уровень технологий, а \(\alpha\) (0 < \(\alpha\) < 1) отражает долю 
капитала в производстве. Эта функция обладает свойством постоянной отдачи от масштаба, что позволяет делить факторы производства на 
«эффективных» работников. Кроме того, в модели принимаются следующие предположения:
\begin{itemize}
  \item Производственная функция имеет \textbf{убывающую отдачу капитала}: при увеличении капитала его предельная продуктивность снижается.
  \item \textbf{Технологический прогресс экзогенен}: темп технологического прогресса \(g\) задан извне и не зависит от решений агентов.
  \item \textbf{Норма сбережений постоянна и экзогенна}: доля выпуска, направляемая на инвестиции, фиксирована.
  \item Рынки характеризуются \textbf{совершенной конкуренцией}, что обеспечивает равенство доходов факторов их предельной продуктивности.
  \item \textbf{Отсутствие adjustment costs}: инвестиции мгновенно трансформируются в новый капитал.
\end{itemize}
Эти предположения упрощают анализ, обеспечивают наличие единственного стационарного равновесия и позволяют предсказать условную конвергенцию 
(стр.~10).

\subsection*{Ответ}
Модель Солоу‑Свана базируется на следующих ключевых предположениях:
\begin{enumerate}
  \item \textbf{Убывающая отдача капитала.} Это означает, что по мере увеличения капитала его предельная продуктивность 
  снижается, что гарантирует существование единственного стационарного уровня капитала.
  \item \textbf{Постоянная отдача от масштаба.} Производственная функция однородна первого порядка, что позволяет анализировать
   рост на душу населения через разделение факторов на «эффективных» работников.
  \item \textbf{Экзогенный технологический прогресс.} Темп технологического прогресса задан извне и является единственным источником 
  долгосрочного роста, поскольку накопление капитала само по себе приводит к убывающей отдаче.
  \item \textbf{Константная норма сбережений.} Фиксированная норма сбережений упрощает динамику накопления капитала и обеспечивает 
  предсказуемость стационарного равновесия.
  \item \textbf{Совершенная конкуренция.} Все агенты (фирмы и домохозяйства) получают вознаграждение в размере предельной продуктивности своих 
  факторов, что устраняет арбитражные возможности.
  \item \textbf{Отсутствие затрат на корректировку капитала.} Это упрощает инвестиционную функцию, позволяя считать, что все инвестиции 
  мгновенно трансформируются в полезный капитал.
\end{enumerate}
Эти предположения важны, поскольку они создают условия для существования единственного стационарного равновесия и позволяют аналитически
 вывести динамику роста, которая предсказывает условную конвергенцию между экономиками с одинаковыми фундаментальными параметрами 
 (стр.~10).
