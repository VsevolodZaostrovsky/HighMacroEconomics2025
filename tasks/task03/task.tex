\section*{Вопрос 3. Почему эндогенный рост производства на душу населения ограничен в модели Солоу‑Свана?}

\subsection*{Контекст}
В модели Солоу‑Свана выпуск на эффективного работника определяется функцией Cobb–Douglas \(y=f(k)\) с убывающей 
отдачей от капитала (\(\partial^2 f/\partial k^2<0\)). Динамика капитала на эффективного работника описывается уравнением
\[
\dot k = s\,f(k) - (n + g + \delta)\,k,
\]
где \(s\) — норма сбережений, \(n\) — темп роста населения, \(g\) — темп технологического прогресса (экзогенен), \(\delta\) — 
амортизация.

\subsection*{Ответ}
Убывающая отдача капитала означает, что по мере роста \(k\) предельный продукт капитала \(f'(k)\) снижается, и инвестиции всё менее 
эффективно превращаются в дополнительный выпуск. Поскольку технологический прогресс экзогенен, модель не содержит 
внутреннего механизма постоянного роста продуктивности. В результате темп роста дохода на душу населения (\(\dot y/y\))
 постепенно падает и в стационарном состоянии стремится к нулю — то есть эндогенный рост ограничен динамикой убывающей отдачи капитала.
