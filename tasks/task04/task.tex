\section*{Вопрос 4. Что такое «золотое» правило сбережения в модели Солоу‑Свана?}

\subsection*{Контекст}
В модели Солоу‑Свана потребление на эффективного работника определяется как разница между выпуском и инвестициями:
\[
c = f(k) - (n + g + \delta)\,k,
\]
где \(f(k)\) — производственная функция Cobb–Douglas, \(n\) — темп роста населения, \(g\) — технологический прогресс (экзогенный), 
\(\delta\) — амортизация.

\subsection*{Ответ}
«Золотое» правило сбережения — это норма сбережений \(s^*\), которая максимизирует потребление на 
эффективного работника в стационарном состоянии. Условие золотого правила:
\[
f'(k^*) = n + g + \delta,
\]
то есть предельная продуктивность капитала равна сумме темпов роста населения, технологического прогресса и амортизации. 
Это обеспечивает наибольшее возможное потребление на душу населения в долгосрочном равновесии.
