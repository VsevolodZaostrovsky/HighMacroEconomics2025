\section*{Вопрос 5. Чем лучше вариант модели с человеческим капиталом?}

\subsection*{Контекст}
В расширенной версии модели Солоу (Mankiw–Romer–Weil, QJE 1992) вводится человеческий капитал \(H\). Производственная функция принимает вид  
\[
F(K,H,L) = K^\alpha H^\beta (AL)^{1-\alpha-\beta},
\]
где \(K\) — физический капитал, \(H\) — человеческий капитал, \(L\) — труд, \(A\) — уровень технологий; \(\alpha,\beta\in(0,1)\), \(\alpha+\beta<1\). Динамика накопления задаётся  
\[
\dot K = s_K K^\alpha H^\beta (AL)^{1-\alpha-\beta} - \delta K,\quad
\dot H = s_H K^\alpha H^\beta (AL)^{1-\alpha-\beta} - \delta H.
\]
(стр.~40) 

\subsection*{Ответ}
Модель с человеческим капиталом превосходит стандартный Солоу‑Свана, поскольку:
\begin{itemize}
  \item устраняет чрезмерную убывающую отдачу от одного фактора, позволяя росту per capita зависеть от накопления как физического, так и человеческого капитала;
  \item лучше объясняет кросс‑страновые различия в доходе через различия в накоплении человеческого капитала;
  \item предсказывает более высокие темпы роста для стран с низким уровнем обоих капиталов (условная конвергенция) и демонстрирует более высокую \(R^2\) в регрессионных тестах;
  \item согласуется с эмпирическими данными лучше стандартной модели Солоу (стр.~40) .
\end{itemize}
