\section*{Вопрос 6. Чем вызвано сглаживание потребления в модели Рамсея–Касса–Купманса?}

\subsection*{Контекст}
Модель Рамсея–Касса–Купманса — это непрерывно‑временная модель межвременного оптимального потребления общества. Социальный 
планировщик максимизирует суммарную дисконтированную полезность
\[
\max_{c(t)\ge0}\;\int_{0}^{\infty} e^{-\beta t} u(c(t))\,dt
\]
при динамике капитала на эффективного работника
\[
\dot k = f(k) - c - (n+g+\delta)\,k,
\]
где \(\beta\) — коэффициент времени предпочтения, \(u(c)\) — строго возрастающая, строго вогнутая
 функция полезности (стр.~50).

Первый порядок условия оптимальности (Эйлерова уравнение) связывает темп роста потребления с разницей между предельной ставкой 
доходности \(r\) и коэффициентом дисконтирования:
\[
\frac{\dot c(t)}{c(t)} = \frac{1}{\theta}\bigl(r(t) - \rho\bigr),
\]
где \(\theta = -u''(c)/u'(c)>0\) — коэффициент межвременной эластичности замещения (стр.~50).

\subsection*{Ответ}
Сглаживание потребления происходит потому, что планировщик равномерно распределяет потребление во времени, стремясь максимизировать 
суммарную полезность при вогнутой функции полезности. Эйлерово уравнение требует, чтобы относительный темп роста потребления был 
пропорционален разнице между рентабельностью капитала \(r\) и субъективной ставкой дисконтирования \(\rho\). Поскольку \(u\) 
строго вогнута (\(\theta>0\)), любое резкое изменение потребления снижает общую полезность, поэтому оптимальный путь 
характеризуется плавным изменением \(c(t)\), а не скачкообразными колебаниями (стр.~50).
