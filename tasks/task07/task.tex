\section*{Вопрос 7. Чем отличаются централизованный и децентрализованный варианты экономики Рамсея?}

\subsection*{Контекст}
Модель Рамсея описывает оптимальное межвременное распределение потребления и накопления капитала.  
\textbf{Централизованный вариант} (социальный планировщик) формулируется как задача максимизации суммарной дисконтированной полезности
\[
\max_{c(t)}\int_{0}^{\infty} e^{-\beta t}u(c(t))dt
\]
при динамике капитала 
\[
\dot k = f(k)-c-(n+g+\delta)k.
\]
\textbf{Децентрализованный вариант} состоит из двух оптимизационных задач:  
\begin{itemize}
  \item Домохозяйство выбирает \(c(t)\) для максимизации собственной полезности, принимая ренту \(r(t)\) и заработную плату \(w(t)\) как данные.
  \item Фирма максимизирует прибыль, нанимая капитал и труд при совершенной конкуренции.
\end{itemize}
Равновесие достигается, когда решения агентов воспроизводят социально оптимальную траекторию (стр.~50).

\subsection*{Ответ}
Главное отличие — в механизме принятия решений:  
\begin{itemize}
  \item В централизованном варианте единый планировщик выбирает \(c(t)\) так, чтобы максимизировать общественное благосостояние, 
  напрямую контролируя накопление капитала.
  \item В децентрализованном варианте оптимизация разбита между домохозяйствами (решают, сколько потреблять и сберегать) и
   фирмами (решают, сколько инвестировать), взаимодействующими через рынок факторов при равновесных ценах \(r(t)\) и \(w(t)\).
\end{itemize}
Несмотря на различие в формулировке, при совершенной конкуренции и отсутствии рыночных искажений оба варианта дают одну и
 ту же динамику потребления и капитала — социально оптимальное равновесие (стр.~50).
