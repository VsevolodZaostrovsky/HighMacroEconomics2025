\section*{Вопрос 8. Определение общего равновесия}

\subsection*{Контекст}
В децентрализованной версии модели Рамсея «общее равновесие» означает такое состояние, при котором одновременно выполняются 
оптимальные решения всех агентов (домохозяйств и фирм) и все рынки (товаров, капитала и труда) сбалансированы. Для модели 
Рамсея это формализуется системой уравнений общего равновесия, включающей:  
\begin{itemize}
  \item Условие оптимального потребления домохозяйства (Эйлерово уравнение).
  \item Условие оптимального выбора капитала фирмой (предельная продуктивность капитала равна ставке доходности \(r\)).
  \item Рыночное равновесие капитала: совокупные сбережения домохозяйств равны инвестициям фирм.
  \item Рыночное равновесие труда: спрос фирм на труд равен предложению.
\end{itemize}
Все эти уравнения совместно определяют динамику \(c(t),k(t),r(t),w(t)\) в экономике в соответствии с социально оптимальным траекторией 
(стр.~50).

\subsection*{Ответ}
Общее равновесие — это состояние, при котором:
\[
\text{(i)}\quad \frac{\dot c}{c}=\frac{1}{\theta}(r-\rho),
\quad
\text{(ii)}\quad r=f'(k),
\quad
\text{(iii)}\quad s\,f(k)=(n+g+\delta)k+c,
\quad
\text{(iv)}\quad \text{рынок труда сбалансирован},
\]
то есть решения агентов оптимальны и все рынки одновременно очищаются. Именно такое состояние 
децентрализованной экономики воспроизводит траекторию социально оптимального (централизованного) решения модели Рамсея.
