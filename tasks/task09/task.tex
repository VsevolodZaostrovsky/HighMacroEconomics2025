\section*{Вопрос 9. Зачем и где нужно условие отсутствия пирамид (no‑Ponzi condition)?}

\subsection*{Контекст}
При оптимизации в непрерывном времени (модель Рамсея–Касса–Купманса или обобщённые модели межвременного потребления) 
агенты могут теоретически наращивать бесконечный долг, финансируя текущие расходы за счёт будущих обязательств.
 Чтобы исключить такие «Понци‑схемы», вводится \emph{условие отсутствия пирамид}:
\[
\lim_{t\to\infty} e^{-\int_0^t r(\tau)\,d\tau}A(t)\;\ge\;0,
\]
где \(A(t)\) — накопленное богатство (стр.~50).

\subsection*{Ответ}
Это условие необходимо, чтобы гарантировать экономическую состоятельность решений: запрещает бесконечное увеличение задолженности и
 требует, чтобы приведённая стоимость активов не была отрицательной в долгосрочном пределе. Без него оптимизационная 
 задача не имела бы смысла, поскольку агент мог бы поддерживать любое текущее потребление, лишь наращивая долг, не 
 заботясь об его погашении. Условие применяется везде, где модели допускают бесконечную временную шкалу и накопление финансовых
  обязательств — в моделях Солоу, Рамсея и OLG — и обеспечивает конечность оптимального пути потребления и капитала (стр.~50).
