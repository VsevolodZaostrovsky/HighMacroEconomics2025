\section*{Вопрос 10. Что искажает искажающее налогообложение и к чему это приводит?}

\subsection*{Контекст}
Искажающее налогообложение в модели Рамсея появляется при финансировании государственных расходов налогами на доход 
от капитала или потребления. В децентрализованном равновесии правительство устанавливает налоговую ставку, влияющую 
на предельную отдачу капитала \(r\) и чистые доходы домашних хозяйств.  

\subsection*{Ответ}
Налог на доход от капитала снижает чистую ставку доходности \(r\), искажая решение домашних хозяйств о сбережениях и инвестициях. 
Это приводит к уменьшению накопления капитала \(k\), снижению долгосрочного уровня выпуска и потребления на душу населения. 
В итоге экономика оказывается в состоянии \emph{динамической неэффективности} по сравнению с социально оптимальным траекторным путём 
(стр.~50).
