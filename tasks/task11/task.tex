\section*{Вопрос 11. Что такое Рикардианская эквивалентность?}

\subsection*{Контекст}
Рикардианская эквивалентность — это теоретический результат в модели межвременного выбора, согласно которому способ финансирования 
государственных расходов (через налоги или долг) не влияет на оптимальное потребление частных домохозяйств. В двухпериодной модели с 
неопределённостью домохозяйство максимизирует
\[
u(c_1) + \frac{1}{1+\rho}E[u(c_2)]
\]
при бюджете \(c_2 = (1 + r_i)(y_1 - c_1) + y_2\). Условие первого порядка даёт
\[
u'(c_1) = \frac{1+r}{1+\rho}E[u'(c_2)].
\]
Если государственный долг невозвратен (no‑Ponzi), то ожидаемая доходность \(E[r_i]\) совпадает с рыночной ставкой \(r\), и 
ковариационный член обнуляется. Тогда
\[
u'(c_1) = \frac{1+r}{1+\rho}E[u'(c_2)],
\]
что означает, что текущее потребление не зависит от способа финансирования дефицита — будь то налоги или выпуск долга (стр.~87–88).

\subsection*{Ответ}
Рикардианская эквивалентность утверждает, что при совершенных рынках и рациональных ожиданиях финансирование дефицита через выпуск
 государственного долга не меняет оптимального потребления домашних хозяйств по сравнению с немедленным налогообложением. 
 Домохозяйства учитывают будущие налоговые обязательства, поэтому текущий дефицит бюджета не влияет на их потребление, 
 а экономическая политика становится нейтральной относительно временного профиля потребления.
