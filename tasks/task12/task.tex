\section*{Вопрос 12. Какие отличия производственной функции в АК‑модели от неоклассической производственной функции? Как понимается капитал?}

\subsection*{Контекст}
АК‑модель — простейкая эндогенная модель роста, в которой предполагается \emph{постоянная отдача от масштаба капитала}. 
Производственная функция имеет линейный вид:
\[
Y = A K,
\]
где \(A\) — константа технологического уровня, \(K\) — агрегированный капитал. В отличие от неоклассической Cobb–Douglas 
функции \(Y=K^\alpha (AL)^{1-\alpha}\) (\(\alpha<1\)), здесь нет убывающей предельной отдачи от капитала (стр.~9).

\subsection*{Ответ}
В АК‑модели производственная функция линейна в капитале, что означает отсутствие убывающей отдачи: каждый дополнительный рубль 
капитала порождает постоянный прирост выпуска. Капитал трактуется агрегированно (включает физический и человеческий капитал, 
технологии и организации), без разграничения на факторы и без амортизации, обеспечивая непрерывный положительный рост дохода 
на душу населения без внешнего технологического прогресса.
