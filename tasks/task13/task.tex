\section*{Вопрос 13. Возможен ли неограниченный эндогенный рост подушевого потребления в АК‑модели? Почему?}

\subsection*{Контекст}
АК‑модель — простейшая эндогенная модель роста, в которой производственная функция линейна в капитале: 
\[
Y = A K,
\]
что означает \emph{постоянную отдачу от капитала} и отсутствие убывающей отдачи. В пер‑capita терминах динамика 
потребления определяется Эйлеровым уравнением
\[
\frac{\dot c}{c} = \frac{f'(k) - (\rho + \delta + n + g)}{\theta},
\]
где \(f'(k)=A\) постоянна (стр.~46).

\subsection*{Ответ}
Да — в АК‑модели per capita потребление может расти неограниченно. Поскольку предельная продуктивность капитала 
\(f'(k)=A\) постоянна и превышает дисконтную ставку \(\rho + \delta + n + g\), темп роста потребления остаётся 
положительным даже при высоком уровне капитала. Отсутствие убывающей отдачи обеспечивает постоянный положительный 
рост per capita без необходимости внешнего технологического прогресса.
