\section*{Вопрос 14. Что такое \(q\)-Тобина?}

\subsection*{Контекст}
В неоклассической теории инвестиций Тобин предложил показатель \(q\), который отражает отношение рыночной стоимости капитала 
к его восстановительной стоимости. В модели с инвестиционными затратами \(q(t)\) определяется как «marginal \(q\)»:
\[
q(t)=\frac{p(t)k(t)+\chi\,\hat I(t)}{\hat k(t)},
\]
где \(p(t)\) — цена капитала, \(\hat k\) — объем капитала, \(\hat I\) — инвестиции, \(\chi\) — коэффициент adjustment cost (стр.~27).

Средний \(q\) (\(\bar q\)) равен дисконтированной сумме ожидаемой предельной продуктивности капитала:
\[
\bar q(t_0)=\frac1{k_0}\int_{t_0}^{\infty} e^{-r(s-t_0)}\bigl(f'(\hat k(s))-\chi'\hat I(s)\bigr)\,ds.
\]

\subsection*{Ответ}
\(q\)-Тобина измеряет привлекательность новых инвестиций:  
\[
q>1\quad\Rightarrow\quad\text{инвестиции приносят прибыль выше издержек},
\]
\[
q<1\quad\Rightarrow\quad\text{инвестиции менее рентабельны и должны сокращаться}.
\]
Таким образом, \(q\) служит сигналом для фирм о целесообразности наращивания капитала и обеспечивает связь рыночных цен 
с реальными инвестиционными решениями.
