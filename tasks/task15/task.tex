\section*{Вопрос 15. Как влияют перераспределительная и накопительная пенсионные системы на равновесие в модели с перерывающимися поколениями?}

\subsection*{Контекст}
В моделях с перекрывающимися поколениями (OLG) рассматриваются механизмы распределения доходов между различными 
поколениями, что существенно влияет на динамику сбережений, накоплений и инвестиций. Два основных типа пенсионных систем:
\begin{itemize}
  \item \textbf{Перераспределительная пенсионная система (PAYG):} в этой системе пенсионные выплаты текущим пенсионерам 
  финансируются за счёт взносов работающего населения. Такой механизм приводит к прямому перераспределению ресурсов от 
  молодого поколения к пожилым и, как следствие, снижает стимулы для самостоятельного накопления капитала. Подобное 
  воздействие подробно анализируется на стр.~86--88.
  \item \textbf{Накопительная пенсионная система:} здесь пенсии формируются на основе индивидуальных сбережений, 
  которые аккумулируются и инвестируются на протяжении трудовой жизни. Это создает дополнительные стимулы для увеличения 
  личных сбережений, что способствует формированию более высокого уровня капитала. В документе приведён анализ влияния 
  накопительной системы на инвестиционную динамику (см. стр.~88--90).
\end{itemize}

\subsection*{Ответ}
\begin{itemize}
  \item \textbf{Перераспределительная пенсионная система (PAYG):} за счёт того, что пенсионные выплаты осуществляются 
  за счет текущих взносов работающего поколения, домохозяйства могут снизить личные сбережения, ожидая получения пенсий. 
  Это приводит к снижению общего уровня накоплений и инвестиций, что в свою очередь ограничивает долгосрочный рост капитала 
  и выпуска на душу населения (см. стр.~86--88).
  
  \item \textbf{Накопительная пенсионная система:} поскольку пенсионные выплаты зависят от накопленных средств, домохозяйства 
  стимулированы к увеличению сбережений. Более высокие сбережения способствуют накоплению капитала и росту инвестиций, что 
  положительно влияет на экономический рост. Однако при чрезмерном накоплении возможны вопросы динамической эффективности, 
  если избыток капитала приводит к снижению предельной отдачи (см. стр.~88--90).
\end{itemize}

Таким образом, выбор между перераспределительной и накопительной пенсионными системами существенно влияет на равновесие в модели с 
перекрывающимися поколениями. Перераспределительная система может снижать стимулы к сбережениям и замедлять рост капитала, тогда
 как накопительная система, напротив, стимулирует накопление и способствует долгосрочному экономическому росту.
