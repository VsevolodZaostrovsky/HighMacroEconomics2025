\section*{Вопрос 16. Что такое динамическая неэффективность?}

\subsection*{Контекст}
Динамическая неэффективность возникает в моделях экономического роста, когда экономика накапливает капитал сверх того 
уровня, при котором потребление на душу населения максимально. При этом предельная отдача капитала \(f'(k)\) оказывается 
ниже суммы темпов роста населения, технологического прогресса и нормы амортизации, то есть выполняется неравенство
\[
f'(k) < n + g + \delta.
\]
В такой ситуации сокращение капитала может привести к повышению текущего потребления, поскольку ресурсы используются не 
оптимально.(см. стр. 51 и стр. 107).

\subsection*{Ответ}
Динамическая неэффективность характеризуется тем, что экономика накапливает капитал в объёмах, превышающих оптимальный 
(золотой) уровень, при котором потребление на душу населения максимально. Если при избытке капитала предельная отдача 
\(f'(k)\) становится ниже суммы темпов роста населения, технологического прогресса и амортизации \((n+g+\delta)\), 
то сокращение капитала ведёт к увеличению текущего потребления без негативных последствий для будущего. Таким образом, 
динамическая неэффективность сигнализирует о том, что избыточные накопления препятствуют достижению максимума социального 
благосостояния, и существуют возможности для улучшения распределения ресурсов (см. стр. 51 и стр. 107).
