\section*{Вопрос 17. Наследование в модели с перерывающими поколениями}

\subsection*{Контекст}
В лекции по модели с перекрывающимися поколениями (overlapping generations model, OLG) рассматривается 
двухпериодный потребитель, который живёт два периода и не имеет возможности передавать богатство потомкам 
(так называемое «нулевое наследство»). В таком базовом OLG‑моделе у домохозяйства второй период жизни заканчивается с 
нулевым уровнем активов: 
\[
a(2)=(1+r)\,a(1)+y(2)-c(2)=0,
\]
что означает отсутствие передачи имущества следующему поколению. Именно такое ограничение («no‑bequest constraint») и
 формализует отсутствие наследования в модели OLG.

\subsection*{Ответ}

В стандартной двухпериодной модели с перекрывающимися поколениями наследование отсутствует: 
в конце второго периода жизни домохозяйство должно иметь нулевой запас активов (bequest \(a(2)=0\)), 
поэтому не может передать богатство потомкам. Это ограничение является частью фундаментальных допущений базовой
 OLG‑модели и означает, что наследование не играет роли в распределении ресурсов между поколениями. 
