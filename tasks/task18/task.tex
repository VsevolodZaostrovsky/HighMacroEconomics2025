\section*{Вопрос 18. Функция денег и супернейтральность в модели с перекрывающимися поколениями}

\subsection*{Контекст}
Модель с перекрывающимися поколениями (OLG) — это базовая двухпериодная макроэкономическая модель, в которой каждое поколение живёт 
два периода: молодость (период труда и заработка дохода) и старость (период потребления без трудового дохода). 
Домохозяйства принимают решения о потреблении и сбережениях, не передавая наследство потомкам (\(a(2)=0\)). В 
такой структуре отсутствуют межпоколенческие передачи богатства, а сбережения молодого поколения финансируют потребление старшего 
поколения через накопления или денежные остатки. Денежная масса \(M(t)\) в модели выступает единственным средством переноса 
покупательной способности во времени: номинальная денежная касса \(m(t)=M(t)/p(t)\) приобретается в молодости и расходуется 
на потребление во старости.


\subsection*{Ответ}
В OLG‑модели функция денег заключается в том, что они являются единственным средством переноса покупательной способности от
 молодости к старости (no‑bequest constraint обеспечивает, что нет альтернативных способов передачи богатства). 
 Денежная касса \(m(t)\) позволяет домохозяйству осуществить потребление во втором периоде жизни, когда трудовой 
 доход отсутствует. Поскольку в базовой модели реальные решения домохозяйств (сбережения, потребление, капитал) 
 зависят исключительно от реальных доходов и ставок процента, изменение денежной массы \(M(t)\) влияет лишь на
  номинальный ценовой уровень \(p(t)\) и не меняет реальных решений — это и есть супернейтральность денег в
   OLG‑модели.

