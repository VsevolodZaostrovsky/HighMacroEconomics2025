\section*{Вопрос 19. Супернейтральность денег в модели Сидрауского. Правило Фридмана}

\subsection*{Контекст}
Модель Сидрауского — это двухпериодная модель перекрывающихся поколений, в которой деньги входят прямо в функцию полезности 
домашнего хозяйства как средство переноса покупательной способности во времени. В формулировке Сидрауского реальная
 денежная касса \(m\) является аргументом функции полезности \(u(c,m)\), при этом предпочтения монетарного агента 
 предполагают отсутствие желания накапливать деньги в долгосрочной перспективе: 
\[
m(t) = \frac{M(t)}{p(t)} = \text{const}, 
\]
а темп инфляции определяется уравнением 
\[
\pi(t) = \frac{p(t+1)}{p(t)} - 1 = -\frac{n}{1+n}.
\]
Супернейтральность денег означает, что изменение денежной массы влияет только на номинальный ценовой уровень, но
 не изменяет реальные решения агентов (потребление, сбережения, капитал).

Правило Фридмана (Chicago rule) в модели Сидрауского устанавливает оптимальный нулевой номинальный процент \(i^*=0\) и, 
следовательно, отрицательную темп инфляции, равную реальному темпу роста экономики \(\pi^*=-n\), чтобы устранить налог 
на держание денег (сеньораж) и минимизировать потери благосостояния. Таким образом, оптимальный денежный рост равен 
темпу прироста населения, что обеспечивает нулевую инфляцию в стационарном равновесии.

\subsection*{Ответ}
Супернейтральность денег в модели Сидрауского проявляется в том, что реальное равновесие (реальные переменные)  
не зависит от уровня денежной массы: деньги лишь переносят покупательную способность между периодами, не влияя
 на реальное потребление или накопление капитала. Оптимальное правило Фридмана требует установить номинальную 
 процентную ставку равной нулю (\(i^*=0\)), что при реальном темпе роста населения \(n\) соответствует темпу
  дефляции \(\pi^*=-n\). Это устраняет налог на держание денег и максимизирует благосостояние агентов. 